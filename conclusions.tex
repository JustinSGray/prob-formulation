\section*{Conclusions}
In this paper, we have presented an approach for describing MDAO problems with constructs and syntax from graph theory. 
Our graph description shares similarities to other approaches such as REMS, $\Psi$, FDT, 
DSM, and XDSM, but it provides new constructs tailored to algorithmic analysis and 
manipulation. The proposed syntax has applications in early phases of MDAO problem 
formulation, especially for problems with large numbers of variables and analyses. 
In particular, we introduce the concepts of the Maximally Connected Graph (MCG) and 
the Fundamental Problem Graph (FPG). 

The MCG addresses the question, ``Given a set of analysis tools, 
what are all of the variable interconnections between them that could be established?''  
The MCG provides a structured formalism to identify \textit{holes} and 
\textit{collisions} in interconnections between the set of analysis tools. 
In order to achieve a valid problem formulation, all of the holes and collisions must 
be resolved. We present an algorithm to perform this resolution which guarantees that an FPG will be found if one exists. In some cases, this algorithm relies on additional input from the 
designers to make decisions. One benefit of a graph theoretic approach is the standard 
algorithms that can be used to inform the user's decisions, such as cycle detection, 
minimum spanning trees, and shortest path algorithms.

The FPG is a graph that describes a data connection structure corresponding 
to an MDAO problem formulation free of collisions and holes. 
An FPG is the result of user choices to fill holes and resolve collisions in 
the initial MCG. Typically, many different FPGs could be attained from a given MCG, 
depending on the user choices. The number of possible FPGs that can be attained by 
selecting different design variables or introducing additional analysis tools to 
fill holes and/or resolve collisions can be viewed as a measure of the freedom 
available to the user in implementing the available analysis tools to 
formulate valid MDAO problems. In Sec.~\ref{s:example problem}, 
we provide an example application of formulating multiple FPGs from an MCG 
for a commercial aircraft design problem. 

For simple problems with few analysis tools and variables, the formulation of a 
valid problem description is straightforward. However, MDAO problems continue to 
grow in scale and complexity. As the numbers of analysis tools and variables have 
increased, it has become increasingly challenging and time consuming for engineering 
teams to determine how the multiple analysis tools can be interconnected to produce 
valid problem formulations, to know when other tools must be introduced, and to 
determine the number of free variables in the problem that can or should be 
varied by the designer or an optimizer. The graph formalism presented in this 
paper is intended to offer value in this context of increasing problem complexity by 
providing a formal approach to systematically identify and resolve missing and 
redundant data in order to create valid MDAO problem formulations.  