\documentclass[]{aiaa-tc} % insert '[draft]' option to show overfull boxes

 \title{An application of Graph Theory to MDAO problem formulation}
        
\author{
  David Pate, %
     \thanks{Georgia Tech}
  Dr. Brian German 
     \thanks{Georgia Tech...}
  Justin Gray,%
     \thanks{Aerospace Engineer, MDAO Branch, Mail Stop 5-11, AIAA Member}   
 }
 
 



%\usepackage{setspace}
%\doublespace

\usepackage{graphicx}
\usepackage{wrapfig}
\usepackage{caption} 
\usepackage{amsmath}
\usepackage{lscape}
\usepackage{hyperref}
\usepackage{appendix}
\usepackage[section]{placeins}

\captionsetup[figure]{margin=5pt,font=small,labelfont=bf,textfont=bf,justification=justified,}
%\captionsetup[wrapfigure]{margin=5pt,font=small,labelfont=bf,justification=justified,singlelinecheck=off}
\captionsetup[table]{margin=5pt,font=small,labelfont=bf,textfont=bf,justification=justified,position=top}

\bibliographystyle{aiaa}

\usepackage{lettrine}
\usepackage{verbatim}

%\usepackage{hyperref} %allows for the creation of actual text links
\begin{document}

\maketitle
 
\begin{abstract}
   blah blah blah
\end{abstract}

\section*{Nomenclature}

\begin{tabular}{l l} 
    AAO      & All-At- \\
    MDAO     & Multidisciplinary Design Analysis and Optimization \\
    FPF      & Fundamental Problem Formulation \\
\end{tabular}


\section{Introduction}
    
    - Large Scale engineered systems 
    - Complex problems that are not obvious to set up
    - need for a formal and consistent way to represent the problem
      - contains *all* the information needed to formulate a solution method
    - Previous attempts to provide problem formulation
 
      \begin{itemize}
        \item Structural Matrix, 
        \item DSM
        \item Functional Dependence Table
        \item XDSM
        \item Flow Chart
        \item Adjacency Matrix
      \end{itemize}

    As the size and complexity of engineering systems grows the time and expense for setting up 
    analysis models grows with them. Multidisciplinary Design Analysis and Optimization (MDAO)
    frameworks such as OpenMDAO\cite{Gray2012} and ModelCenter have enabled a new level of analysis tool integration 
    and paved the way for models of with more analysis tools and increasing numbers of multidisciplinary couplings. 
    Such large models present a distinct challenge to properly set up in a manner that allows for an efficient solution. 
    In fact as the size of a problem grows very large just determining what the proper solution strategy can be a daunting 
    task. 

    What is needed, in order to address the complexity problem, is a method of specifying an engineering problem 
    in a complete and totally general manner. A complete specification needs to include the following information: 
    \begin{itemize}
       \item Discipline Analyses 
       \item Discipline State Variables and Residuals
       \item Local \& Global Design Variables
       \item Local \& Global Constraints
       \item Objective or Objectives
       \item Coupling Constraints
       \item Local and Global Parameters
    \end{itemize}
    A general specification is one that states nothing specific about a problem solution path. It does not include any details about 
    how an objective is to be minimized, how coupling constraints will be satisfied, or how residuals will managed. We 
    define a complete and general problem specification to be the Fundamental Problem Formulation (FPF). By definition, the FPF for 
    any given problem will be constant regardless of which MDAO framework or optimization architecture is used to solve the problem. 

    In this work we proposed a graph based syntax for the specification of the FPF. This graph based syntax provides several key
    features that make it useful for working with large scale MDAO problems. It provides a rigid structure that can be easily manipulated 
    in with a wide range of well establish graph-theory algorithms. By taking advantage of that structure it is possible to test 
    any given problem specification to determine if it is in fact complete and general. It can be used to describe more specific problem 
    setups, and given one can be used to derive the FPF from it. Lastly, a graph based specification for problem formulation lends itself
    well to interacting with MDAO frameworks which makes it particularly powerful for integrating problem formulation analysis techniques 
    into those frameworks. 


\section{Problem Formulation Syntax}
    The most common problem formulation for an MDAO problem could be given as follows: 

    \begin{align}
        min &\ \ f\left(X\right) \notag
        \\ w.r.t. &\ \  X \notag
        \\ s.t. &\ \ G\left(X\right) \leq 0
        \\      &\ \ C\left(X\right) \leq 0
    \end{align}

    Where $f(X)$ is the objective function, $X$ is the vector of design variables, $G(X)$ are the optimization constraints, 
    and $C(X)$ are the coupling constraints. This does represent a general problem formulation, since it says nothing about 
    how the problem should be solved. Tedford and Martins used this format to specify a set of test problems and derived specific formulations 
    for solving them with a number of optimization architectures\cite{Tedford2009}. Lamb and Martins also used this basic mathematical 
    syntax to describe the All-At-Once (AAO) architecture \cite{Lambe2011}. Although the AAO specification does specify a solution path 
    including a single optimizer to solve the entire problem, in every other way it fills the criterion for a general problem specification. 

    The challenge with using this traditional mathematical syntax is that it is not easily manipulated or analyzed algorithmically. 
    A number of matrix based methods have been used successfully to manipulate problem formulations in a programmatic manner. Steward's 
    Design Structure Matrix (DSM) is a form of an adjacency matrix which captures the relationship between design elements and 
    design variables where off diagonal elments of the matrix indicate coupling\cite{Steward1981}. 
    Although it does not have to provide a specific solution path, the DSM has been successfully 
    applied to allow selection of an optimal ordering for analysis tools in a complex system. Rogers et. al developed DeMAID to manipulate a
    DSM to minimizing the computational costs of solving highly coupled systems\cite{Rogers1996}. 
    Since these matrix based methods operate on some form of an adjacency matrix, they can all be represented in an equivalent graph. 



\section{Graph Based Problem Formulation}

\section{Example Problem}

\section{Applications}

\section{Conclusions}

\end{document}