% IDF architecture diagram produced by the TikZ package
\documentclass{article}
\usepackage{geometry}
\usepackage{amsfonts}
\usepackage{amsmath}
\usepackage{amssymb}

\usepackage{tikz}

% Define the set of tikz packages to be included in the architecture diagram document

\usetikzlibrary{arrows,chains,positioning,scopes,shapes.geometric,shapes.misc,shadows} 

% Set the border around all of the architecture diagrams to be tight to the diagrams themselves
% (i.e. no longer need to tinker with page size parameters)

\usepackage[active,tightpage]{preview}
\PreviewEnvironment{tikzpicture}
\setlength{\PreviewBorder}{5pt}


\begin{document}

% Style definition outside any picture
% Define all the styles used to produce XDSMs for MDO

% Component types
\tikzstyle{Optimization} = [rounded rectangle,draw,fill=blue!20,inner sep=6pt,minimum height=1cm,text badly centered]
\tikzstyle{LP_Optimization} = [rectangle,draw,fill=blue!20,inner sep=6pt,minimum height=1cm,text badly centered]
\tikzstyle{Analysis} = [rectangle,draw,fill=green!20,inner sep=6pt,minimum height=1cm,text badly centered]
\tikzstyle{Function} = [rectangle,draw,fill=purple!20,inner sep=6pt,minimum height=1cm,text badly centered]
\tikzstyle{MDA} = [rounded rectangle,draw,fill=orange!20,inner sep=6pt,minimum height=1cm,text badly centered]
\tikzstyle{Metamodel} = [rectangle,draw,fill=yellow!20,inner sep=6pt,minimum height=1cm,text badly centered]
\tikzstyle{DOE} = [rounded rectangle,draw,fill=yellow!20,inner sep=6pt,minimum height=1cm,text badly centered]
%\tikzstyle{OptFunction} = [rectangle,draw,fill=red!20,inner sep=6pt,minimum height=1cm,text badly centered]

%% A simple command to give the repeated structure look for components and data
\tikzstyle{stack} = [double copy shadow]

%% A simple command to fade components and data, e.g. demonstrating a sequence of steps in an animation
\tikzstyle{faded} = [draw=black!50,fill=white,text opacity=0.5]

%% Simple fading commands for the lines
\tikzstyle{fadeddata} = [color=black!20]
\tikzstyle{fadedprocess} = [color=black!50]

% Component types for repeated structures (i.e. for parallel structures)
\tikzstyle{Optimization_i} = [double copy shadow, Optimization]
\tikzstyle{LP_Optimization_i} = [double copy shadow, LP_Optimization]
\tikzstyle{Analysis_i} = [double copy shadow, Analysis]
\tikzstyle{Function_i} = [double copy shadow, Function]
\tikzstyle{MDA_i} = [double copy shadow, MDA]
\tikzstyle{Metamodel_i} = [double copy shadow, Metamodel]
\tikzstyle{DOE_i} = [double copy shadow, DOE]

% Faded component types for, e.g. demonstrations of each step. We use these style definitions to "gray out" large parts of the diagram.
\tikzstyle{Optimization_fade} = [Optimization,fill=blue!10,draw=black!30,text opacity=0.3]
\tikzstyle{Analysis_fade} = [Analysis,fill=green!10,draw=black!30,text opacity=0.3]
\tikzstyle{Function_fade} = [Function,fill=purple!10,draw=black!30,text opacity=0.3]
\tikzstyle{MDA_fade} = [MDA,fill=orange!10,draw=black!30,text opacity=0.3]
\tikzstyle{Metamodel_fade} = [Metamodel,fill=yellow!10,draw=black!30,text opacity=0.3]
\tikzstyle{DOE_fade} = [DOE,fill=yellow!10,draw=black!30,text opacity=0.3]

\tikzstyle{Optimization_i_fade} = [Optimization_i,fill=blue!10,draw=black!30,text opacity=0.3]
\tikzstyle{Analysis_i_fade} = [Analysis_i,fill=green!10,draw=black!30,text opacity=0.3]
\tikzstyle{Function_i_fade} = [Function_i,fill=purple!10,draw=black!30,text opacity=0.3]
\tikzstyle{MDA_i_fade} = [MDA_i,fill=orange!10,draw=black!30,text opacity=0.3]
\tikzstyle{Metamodel_i_fade} = [Metamodel_i,fill=yellow!10,draw=black!30,text opacity=0.3]
\tikzstyle{DOE_i_fade} = [DOE_i,fill=yellow!10,draw=black!30,text opacity=0.3]

% Data types
\tikzstyle{DataInter} = [trapezium,trapezium left angle=75,trapezium right angle=105,draw,fill=black!10]
\tikzstyle{DataIO} = [trapezium,trapezium left angle=75,trapezium right angle=105,draw,fill=white]

% Data types for repeated structures
\tikzstyle{DataInter_i} = [double copy shadow, DataInter]
\tikzstyle{DataIO_i} = [double copy shadow, DataIO]

% Faded data types
\tikzstyle{DataInter_fade} = [DataInter,draw=black!30,fill=white,text opacity=0.3]
\tikzstyle{DataIO_fade} = [DataIO_i,draw=black!30,fill=white,text opacity=0.3]

\tikzstyle{DataInter_i_fade} = [DataInter_i,draw=black!30,fill=white,text opacity=0.3]
\tikzstyle{DataIO_i_fade} = [DataIO_i,draw=black!30,fill=white,text opacity=0.3]

% Edges
\tikzstyle{DataLine} = [color=black!40,line width=5pt]
\tikzstyle{ProcessHV} = [-,line width=1pt,to path={-| (\tikztotarget)}]
\tikzstyle{ProcessTip} = [-,line width=1pt]

% Faded edges
\tikzstyle{DataLine_fade} = [DataLine,color=black!10]
\tikzstyle{ProcessHV_fade} = [ProcessHV,color=black!30]
\tikzstyle{ProcessTip_fade} = [ProcessTip,color=black!30]

% Matrix options
\tikzstyle{MatrixSetup} = [row sep=3mm, column sep=2mm]

% Declare a background layer for showing node connections
\pgfdeclarelayer{data}
\pgfdeclarelayer{process}
\pgfsetlayers{data,process,main}

% A new command to split the component text over multiple lines
\newcommand{\MultilineComponent}[3]
{
	\begin{minipage}{#1}
	\begin{center}
		#2 \linebreak #3
	\end{center}
	\end{minipage}
}

% A new command to split the component text over multiple columns
\newcommand{\MultiColumnComponent}[5]
{
	\begin{minipage}{#1}
	\begin{center}
	#2 \linebreak #3
	\end{center}
	\begin{minipage}{0.49\textwidth}
	\begin{center}
	#4
	\end{center}
	\end{minipage}
	\begin{minipage}{0.49\textwidth}
	\begin{center}
		#5
	\end{center}
	\end{minipage}
	\end{minipage}
}


\begin{tikzpicture}

	% Use a matrix to line up all the nodes 
    \matrix[MatrixSetup]{
		% First row
		& \node [DataIO](input_x){$x_1^{(0)},z_1^{(0)},z_2^{(0)}$};
		&
		& \node [DataInter](input_y){$\hat{y_2}^{(0)}$};
		&
		& 
		&
		\\
		% second row
		\node [DataIO] (x_output) {$x_1^*,z_1^*,z_2^*$};
		& \node [Optimization] (opt) {\MultilineComponent{2.3cm}{0,7$\rightarrow$1:}{Optimization}};
		& 
		& \node [DataInter](opt-x1){$1:x_1,z_1,z_2$};
		& \node [DataInter](opt-x2){$1:z_1,z_2$};
		& 
		& \node [DataInter](opt-x3){$1:x_1,z_2$};
		\\
		% third row
		& 
		& \node [MDA] (mda) {\MultilineComponent{1.5cm}{1, 5$\rightarrow$2:}{MDA}};
		& \node [DataInter](mda-y2){$2:\hat{y_2}$};
		& 
		& \node [DataInter](mda-y2_2){$2:\hat{y_2}$};
		& 
		\\
		% fourth row
		\node [DataIO] (y_output) {$y_1^*$};
		& 
		& 
		& \node [Analysis] (D_1) {\MultilineComponent{1.6cm}{2:}{$D_1$}};
		& \node [DataInter](mda-y1){$3:y_1$};
		& 
		& \node [DataInter](mdadone-y1){$6:y_1$};
		\\
		% fifth row
		\node [DataIO] (y2_output) {$y_2^{*}$};
		& 
		& 
		& 
		& \node [Analysis] (D_2) {\MultilineComponent{1.6cm}{3:}{$D_2$}};
		& \node [DataInter](mda-y2){$4:y_2$};
		& \node [DataInter](mdadone-y2){$6:y_2$};
		\\
		% sixth row
		& 
		& \node [DataInter](mda-g){$5:h$};
		& 
		& 
		& \node [Function] (constraints) {\MultilineComponent{1.6cm}{4:}{H}}; 
		& 
		\\
		% seventh row
		& \node [DataInter](opt-f){$7:f,g_1,g_2$};
		& 
		& 
		& 
		& 
		& \node [Function_i] (objective) {\MultilineComponent{1.6cm}{6:}{F,$G_1$,$G_2$}}; 
		\\
	};
	
	% Now define edges to tie the nodes together, using chains	
	{ [start chain=process]
		\begin{pgfonlayer}{process}
		\chainin (input_x);
		\chainin (opt)		[join=by ProcessTip];
		\chainin (mda)		[join=by ProcessHV];
		\chainin (D_1)		[join=by ProcessHV];
		\chainin (D_2)		[join=by ProcessHV];
		\chainin (constraints)		[join=by ProcessHV];
		\chainin (mda)		[join=by ProcessHV];
		\chainin (objective)	[join=by ProcessHV];
		\chainin (opt) [join=by ProcessHV];
		\chainin (x_output)	[join=by ProcessTip];
		\end{pgfonlayer}
	}
	
	\begin{pgfonlayer}{data}
	% Vertical edges
	\path (input_x) edge [DataLine] (opt-f)
	(input_y) edge [DataLine] (D_1)
	(input_x) edge [DataLine] (opt)
	(opt) edge [DataLine] (opt-x1)
	(opt) edge [DataLine] (opt-x2)
	(opt) edge [DataLine] (opt-x3)
	(D_1) edge [DataLine] (mda-y1)
	(D_1) edge [DataLine] (mdadone-y1)
	(D_2) edge [DataLine] (mda-y2)
	(D_2) edge [DataLine] (mdadone-y2)
	(D_2) edge [DataLine] (opt-x2)
	(constraints) edge [DataLine] (mda-g)
	(constraints) edge [DataLine] (mda-y2)
	(constraints) edge [DataLine] (mda-y2_2)
	(mda-g) edge [DataLine] (mda)
	(objective) edge [DataLine] (opt-f)
	(objective) edge [DataLine] (opt-x3)

	(y_output) edge [DataLine] (D_1)
	(y2_output) edge [DataLine] (D_2)
;
	\end{pgfonlayer}
	
\end{tikzpicture}

\end{document}